%-------------------------------------
% LaTeX Resume for Graduate Recommendation (CN)
% Original Author : Leslie Cheng
% Modified by : Levitate Qian
% License : MIT
%-------------------------------------

\documentclass[letterpaper,12pt]{article}[leftmargin=*]
\usepackage{xeCJK}
\usepackage[empty]{fullpage}
\usepackage{enumitem}
\usepackage{ifxetex}
\ifxetex
  \usepackage{fontspec}
  \usepackage[xetex]{hyperref}
\else
  \usepackage[utf8]{inputenc}
  \usepackage[T1]{fontenc}
  \usepackage[pdftex]{hyperref}
\fi
\usepackage{fontawesome5}
\usepackage[sfdefault,light]{FiraSans}
% \usepackage{anyfontsize}
\usepackage{xcolor}
\usepackage{tabularx}
\usepackage{graphicx}
\usepackage{multirow}
\usepackage{multicol}

% 定义字体,请记得从网站提供的地址下载
\setCJKsansfont{思源黑体 CN Light}[
    Path = ./fonts/, 
    UprightFont = SOURCEHANSANSCN-LIGHT.OTF, % 直立体
    BoldFont = SOURCEHANSANSCN-BOLD.OTF,       % 粗体
    ItalicFont = JiangChengXieHei 200W.ttf,    % 斜体
] 
\newcommand{\semibook}{\firabook \CJKfontspec[Path = ./fonts/]{SOURCEHANSANSCN-REGULAR.OTF} }

% 如果你没有下载字体,请取消注释下列命令,以尝试初次编译能否成功
% \newcommand{\semibook}{\firabook}

%-------------------------------------------------- SETTINGS HERE --------------------------------------------------
% Header settings
\def \fullname {XXX}
\def \subtitle {XX大学XXXX级本科生}

\def \phoneicon {\faPhone*}
\def \phonetext {XXX~XXXX~XXXX}

\def \hometownicon {\faMapMarker}
\def \hometowntext {XX省XX市}

\def \birthdayicon {\faBirthdayCake}
\def \birthdaytext {20XX.XX.XX}

\def \emailicon {\faEnvelope}
\def \emaillink {mailto:XXXX@outlook.com}
\def \emailtext {XXXX@outlook.com}

% \def \githubicon {\faGithub}
% \def \githublink {https://github.com/Levitate-Qian}
% \def \githubtext {/Levitate-Qian}

\def \websiteicon {\faGlobeAsia}
\def \websitelink {https://levitate-qian.github.io/}
\def \websitetext {levitate-qian.github.io/}


\def \headertype {\singlecol} % \singlecol or \doublecol % 双栏并未测试

% Misc settings
\def \entryspacing {-0pt}

\def \bulletstyle {\faAngleRight}

% Define colours
\definecolor{primary}{HTML}{000000}
\definecolor{secondary}{HTML}{0D47A1}
% \definecolor{secondary}{HTML}{b0252a} %西电红
% \definecolor{secondary}{HTML}{004181} %西电蓝
\definecolor{accent}{HTML}{263238}
\definecolor{links}{HTML}{1565C0}
% \definecolor{links}{HTML}{bf5b5b} %西电红(浅)


%------------------------------------------------------------------------------------------------------------------- 

% Defines to make listing easier
\def \linkedin {\linkedinicon \hspace{3pt}\href{\linkedinlink}{\linkedintext}}
\def \phone {\phonetext}
\def \hometown {\hometowntext}
\def \birthday {\birthdaytext}
\def \email {\href{\emaillink}{\emailtext}}
\def \github {\githubicon \hspace{3pt}\href{\githublink}{\githubtext}}
\def \website {\href{\websitelink}{\websitetext}}

% Adjust margins
\addtolength{\oddsidemargin}{-0.55in}
\addtolength{\evensidemargin}{-0.55in}
\addtolength{\textwidth}{1.1in}
\addtolength{\topmargin}{-0.6in}
\addtolength{\textheight}{1.1in}

% Define the link colours
\hypersetup{
    colorlinks=true,
    urlcolor=links,
}

% Set the margin alignment 
\raggedbottom
\raggedright
\setlength{\tabcolsep}{0in}

%-------------------------
% Custom commands

% Sections
\renewcommand{\section}[3]{\vspace{5pt}
  \colorbox{secondary}{\color{white}\raggedbottom\normalsize\textbf{{#1}{\hspace{7pt}#2}}}{#3}
}

% Entry start and end, for spacing
\newcommand{\resumeEntryStart}{\begin{itemize}[leftmargin=2.5mm]}
\newcommand{\resumeEntryEnd}{\end{itemize}\vspace{\entryspacing}}

% Itemized list for the bullet points under an entry, if necessary
\newcommand{\resumeItemListStart}{\begin{itemize}[leftmargin=4.5mm,rightmargin=2.5mm]}
\newcommand{\resumeItemListEnd}{\end{itemize}}

% Resume item
\renewcommand{\labelitemii}{\bulletstyle}
\newcommand{\resumeItem}[1]{
  \item\small{
    {#1 \vspace{-2pt}}
  }
}

% Entry with title, subheading, date(s), and location
\newcommand{\resumeEntryTSDL}[4]{
  \vspace{-1pt}\item[]
    \begin{tabularx}{0.97\textwidth}{X@{\hspace{60pt}}r}
      \textbf{\color{secondary}#1} & {\semibook\color{accent}\small#2} \\
      \textit{\color{accent}\small#3} & \textit{\color{accent}\small#4} \\
    \end{tabularx}\vspace{-6pt}
}

% Entry with title and date(s)
\newcommand{\resumeEntryTD}[2]{
  \vspace{-1pt}\item[]
    \begin{tabularx}{0.97\textwidth}{X@{\hspace{60pt}}r}
      \textbf{\color{secondary}#1} & {\semibook\color{accent}\small#2} \\
    \end{tabularx}\vspace{-6pt}
}

% Entry for special (skills)
\newcommand{\resumeEntryS}[2]{
  \item[]\small{
    \textbf{\color{secondary}#1 }{ #2 }
  }
}

% Entry for special (Awards)
\newcommand{\resumeEntryA}[2]{
  \item\small{{\par #1 \hfill #2 \par\vspace{-2pt}}%
  }
}

% Double column header
\newcommand{\doublecol}[6]{
  \begin{tabularx}{\textwidth}{Xr}
    {
      \begin{tabular}[c]{l}
        \fontsize{35}{45}\selectfont{\color{primary}{{\textbf{\fullname}}}} \\
        {\textit{\subtitle}} % You could add a subtitle here
      \end{tabular}
    } & {
      \begin{tabular}[c]{l@{\hspace{1.5em}}ll}
        {\small#4} & {\small#1} &\multirow{3}*{\includegraphics[width=0.75in]{photo.jpg}}\\
        {\small#5} & {\small#2} \\
        {\small#6} & {\small#3}
      \end{tabular}
    }
  \end{tabularx}
}

% Single column header
\newcommand{\singlecol}[5]{
  \begin{tabularx}{0.985\textwidth}{Xr}
    {
      \begin{tabular}[b]{l}
        \fontsize{35}{45}\selectfont{\color{secondary}{{\textbf{\fullname}}}} \vspace{15pt}\\
        {\textit{\subtitle}}   \vspace{5pt}% You could add a subtitle here
      \end{tabular}
    } & {
      \begin{tabular}[b]{c@{\hspace{5pt}}l@{\hspace{1.5em}}l}
        {\phoneicon} &{\small#1} &\multirow{3}*{\includegraphics[width=0.75in]{photo.jpg}}\\
        {\emailicon} &{\small#2} \\
        {\websiteicon} &{\small#3} \\
        {\hometownicon} &{\small#4} \\
        {\birthdayicon} &{\small#5} 
      \end{tabular}
    }
  \end{tabularx}
}

\begin{document}
%-------------------------------------------------- BEGIN HERE --------------------------------------------------

%---------------------------------------------------- HEADER ----------------------------------------------------

\headertype{\phone}{\email}{\website}{\hometown}{\birthday} % Set the order of items here
\vspace{-10pt} % Set a negative value to push the body up, and the opposite

%-------------------------------------------------- EDUCATION --------------------------------------------------
\section{\faGraduationCap}{Education}{}

  \resumeEntryStart
    \resumeEntryTSDL
      {西安电子科技大学}{20XX.09 -- 至今}
      { 电子工程学院,电子信息工程{\footnotesize (不区分方向)},担任XXXXXXXX。}{西安,陕西}
      \resumeItemListStart
        \resumeItem {前五学期推免排名课程均分{XX.XX},{\semibook 排名XX/XXX}。}
        \resumeItem {{\semibook 前六学期推免排名课程均分{XX.XX}},必修与限选均分XX.XX,GPA为3.9/4.0。主干课程中,信号与系统(XX),随机信号分析(XX),数字信号处理(XX)。}
      \resumeItemListEnd
  \resumeEntryEnd


%-------------------------------------------------- AWARDS --------------------------------------------------
\section{\faTrophy}{Awards}{{\footnotesize\color{accent}\textmd{\textit{ (访问\href{https://levitate-qian.github.io/details/}{博客详情}了解更多,访问密码为XXXXXXXXXXXX)}}}}

  \resumeEntryStart
    \resumeEntryTD
      {奖励荣誉}{\color{secondary}\faCalendar}
    \resumeItemListStart
    \resumeEntryA{{\semibook 20XX~-~20XX学年国家奖学金、校级优秀学生标兵}}{20XX.11}
    \resumeEntryA{20XX~-~20XX学年校级一等奖学金、校级优秀学生}{20XX.11}
    \resumeItemListEnd

    \resumeEntryTD
      {学科竞赛}{\color{secondary}\faCalendar}
      \resumeItemListStart
      \resumeEntryA{{\semibook 20XX年国际大学生数学建模竞赛XXXXX}}{20XX.04}
      \textit{\color{accent}\fontsize{9pt}{1}\selectfont 这个奖有什么特别之处,比如比例、报道等。}
      \resumeEntryA{20XX年全国大学生数学建模竞赛XXXXX}{20XX.12}
      \resumeEntryA{20XX年全国大学生数学竞赛(非数学类)XXXXX}{20XX.11}
      \resumeItemListEnd
  \resumeEntryEnd

  % \resumeEntryStart
  %   \resumeEntryTSDL
  %     {Staples}{Mar. 2008 -- Mar. 2008}
  %     {Sales Associate}{Scranton, PA}
  %   \resumeItemListStart
  %       \resumeItem {Became the top salesman of the store within a one-month timespan}
  %       \resumeItem {Made a record-high sales figure despite having an unfunny boss}
  %       \resumeItem {Provided extraordinary and exceptional customer service to the masses}
  %   \resumeItemListEnd
  % \resumeEntryEnd

  % \resumeEntryStart
  %   \resumeEntryTSDL
  %     {Dunder Mifflin}{Mar. 2005 -- Mar. 2008}
  %     {Assistant (to the) Regional Manager}{Scranton, PA}
  %   \resumeItemListStart
  %     \resumeItem {Acted as Regional Manager's eyes, ears, and right hand, overseeing and reporting on employee conduct}
  %     \resumeItem {Provided services to the office such as martial arts and surveillance}
  %     \resumeItem {Introduced new linen paper lines into the market, often closing sight-unseen sales}
  %   \resumeItemListEnd
  % \resumeEntryEnd

    %-------------------------------------------------- PROGRAMMING SKILLS --------------------------------------------------
\section{\faBuffer}{Skills}{}
\resumeEntryStart
\resumeEntryS{英语水平 } {在……做了……(用于支撑英语水平的案例)。{\semibook 国家英语四、六级分别为XXX与XXX分。}}
\resumeEntryS{编程水平 } {{\semibook 熟悉……(语言)。}在……中,利用……,实现了……。在……中,利用……,实现了……。在……中,利用……,实现了……。(用于支撑编程水平的案例)}
\resumeEntryS{XXXX } {{\semibook …………………………。}在哪儿做什么,有什么收获。(用于支撑该技能的案例)}
\resumeEntryEnd

%-------------------------------------------------- PROJECTS --------------------------------------------------
\section{\faFlask}{Projects}{}

  \resumeEntryStart

\resumeEntryTD
  {竞赛 | 20XX年国际大学生数学建模竞赛}{XXXXXX AWARD \& XXXXXX Winner }
\resumeItemListStart
  \resumeItem {你做了什么?}
  \resumeItem {在竞赛中做了什么?}
\resumeItemListEnd


  \resumeEntryTD
  {项目 | XXXX}{(指导老师名字)}
\resumeItemListStart
  \resumeItem {你做了什么?}
  \resumeItem {该项目是做什么的?}
\resumeItemListEnd
% \resumeEntryEnd

\resumeEntryTD
{论文 | XXXX}{(指导老师名字)}
\resumeItemListStart
\resumeItem {你做了什么?}
\resumeItem {该论文是做什么的?有什么创新之处?收获了什么?}
\resumeItemListEnd

  \resumeEntryEnd



\end{document}
