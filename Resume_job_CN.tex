%-------------------------------------
% LaTeX Resume for Job (CN)
% Original Author : Leslie Cheng
% Modified by : Levitate Qian
% License : MIT
%-------------------------------------

\documentclass[a4paper,12pt]{article}[leftmargin=*]
\usepackage{xeCJK}
\usepackage[empty]{fullpage}
\usepackage{enumitem}
\usepackage{ifxetex}
\ifxetex
  \usepackage{fontspec}
  \usepackage[xetex]{hyperref}
\else
  \usepackage[utf8]{inputenc}
  \usepackage[T1]{fontenc}
  \usepackage[pdftex]{hyperref}
\fi
\usepackage{fontawesome5}
\usepackage[sfdefault,light]{FiraSans}
\usepackage{anyfontsize}
\usepackage{xcolor}
\usepackage{tabularx}
\usepackage{graphicx}
\usepackage{multirow}
\usepackage{multicol}
\usepackage{fancybox}
\usepackage{ragged2e}

% 定义字体,请记得从网站提供的地址下载
\setCJKsansfont{思源黑体 CN Light}[
    Path = ./fonts/, 
    UprightFont = SOURCEHANSANSCN-LIGHT.OTF, % 直立体
    BoldFont = SOURCEHANSANSCN-BOLD.OTF,       % 粗体
    ItalicFont = JiangChengXieHei 200W.ttf,    % 斜体
] 
\newcommand{\semibook}{\firabook \CJKfontspec[Path = ./fonts/]{SOURCEHANSANSCN-REGULAR.OTF} }
\newcommand{\semibf}{\firamedium \CJKfontspec[Path = ./fonts/]{SOURCEHANSANSCN-MEDIUM.OTF} }
\newcommand{\semibookit}{\itshape\firabook \CJKfontspec[Path = ./fonts/]{JiangChengXieHei 400W.TTF} }
\newcommand{\semibfit}{\itshape\firamedium \CJKfontspec[Path = ./fonts/]{JiangChengXieHei 500W.TTF} }
\newcommand{\heavybf}{\firaheavy \CJKfontspec[Path = ./fonts/]{SOURCEHANSANSCN-HEAVY.OTF} }


% 如果你没有下载字体,请取消注释下列命令,以尝试初次编译能否成功
% \newcommand{\semibook}{\firabook}
% \newcommand{\semibf}{\firamedium}
% \newcommand{\semibookit}{\itshape\firabook }
% \newcommand{\semibfit}{\itshape\firamedium }
% \newcommand{\heavybf}{\firaheavy}

% 如果你本地已经安装了上述字体,可以转而使用下述命令
% \setCJKsansfont[ItalicFont={江城斜黑体 200W}]{思源黑体 CN Light}
% \newcommand{\semibook}{\firabook \CJKfontspec{思源黑体 CN} }
% \newcommand{\semibf}{\firamedium \CJKfontspec{思源黑体 CN Medium} }
% \newcommand{\semibookit}{\itshape\firabook \CJKfontspec{江城斜黑体 400W} }
% \newcommand{\semibfit}{\itshape\firamedium \CJKfontspec{江城斜黑体 500W} }
% \newcommand{\heavybf}{\firaheavy \CJKfontspec{思源黑体 CN Heavy}}


%-------------------------------------------------- SETTINGS HERE --------------------------------------------------
% Header settings
\def \fullname {XXX}
\def \subtitle {XX大学20XX届硕士研究生}
% \def \subtitle {求职意向:}


\def \phoneicon {\faPhone*}
\def \phonetext {XXX~XXXX~XXXX}

\def \hometownicon {\faHome}
\def \hometowntext {XX省XX市}

\def \birthdayicon {\faBirthdayCake}
\def \birthdaytext {20XX.XX.XX [XX岁]}

\def \emailicon {\faEnvelope}
\def \emaillink {mailto:XXXX@outlook.com}
\def \emailtext {XXXX@outlook.com}

% \def \githubicon {\faGithub}
% \def \githublink {https://github.com/Levitate-Qian}
% \def \githubtext {/Levitate-Qian}

\def \websiteicon {\faGlobeAsia}
\def \websitelink {https://levitate-qian.github.io/}
\def \websitetext {levitate-qian.github.io/}


\def \headertype {\singlecol} % \singlecol or \doublecol % 双栏并未测试

% Misc settings
\def \entryspacing {-0pt}

\def \bulletstyle {\faAngleRight}

% Define colours
\definecolor{primary}{HTML}{000000}
\definecolor{secondary}{HTML}{0D47A1}
% \definecolor{secondary}{HTML}{b0252a} %西电红
% \definecolor{secondary}{HTML}{004181} %西电蓝
\definecolor{accent}{HTML}{263238}
\definecolor{links}{HTML}{1565C0}
% \definecolor{links}{HTML}{bf5b5b} %西电红(浅)


%------------------------------------------------------------------------------------------------------------------- 

% Defines to make listing easier
\def \linkedin {\linkedinicon \hspace{3pt}\href{\linkedinlink}{\linkedintext}}
\def \phone {\phonetext}
\def \hometown {\hometowntext}
\def \birthday {\birthdaytext}
\def \email {\href{\emaillink}{\emailtext}}
\def \github {\githubicon \hspace{3pt}\href{\githublink}{\githubtext}}
\def \website {\href{\websitelink}{\websitetext}}

% Adjust margins
\addtolength{\oddsidemargin}{-0.55in}
\addtolength{\evensidemargin}{-0.55in}
\addtolength{\textwidth}{1.1in}
\addtolength{\topmargin}{-0.6in}
\addtolength{\textheight}{1.1in}

% Define the link colours
\hypersetup{
    colorlinks=true,
    urlcolor=links,
}

% Set the margin alignment 
\raggedbottom
\raggedright
% \RaggedRight
\setlength{\tabcolsep}{0in}

%-------------------------
% Custom commands

% 这个定义了两个button,目前设置的是button显示,buttonb不显示,可以视自己需求使用
\newcommand{\button}[1]{
  {\color{secondary}\scriptsize\mdseries\ovalbox{#1}}\hspace{1pt}
}
\newcommand{\buttonb}[1]{
  % {\color{secondary}\footnotesize\mdseries\ovalbox{#1}}\hspace{5pt}
}

% Sections
\renewcommand{\section}[3]{\vspace{5pt}
  \colorbox{secondary}{\color{white}\normalsize\textbf{{#1}{\hspace{5pt}{\color{secondary}\rule[-2pt]{1pt}{10pt}}#2}}}#3
}

% Entry start and end, for spacing
\newcommand{\resumeEntryStart}{\begin{itemize}[leftmargin=2.5mm]}
\newcommand{\resumeEntryEnd}{\end{itemize}\vspace{\entryspacing}}

% Itemized list for the bullet points under an entry, if necessary
\newcommand{\resumeItemListStart}{\begin{itemize}[leftmargin=4.5mm,rightmargin=2.5mm]}
\newcommand{\resumeItemListEnd}{\end{itemize}}

% Itemized list for the bullet points under an entry, if necessary
\newcommand{\resumeItemListStartP}{\begin{itemize}[leftmargin=7mm,rightmargin=2.5mm]}
\newcommand{\resumeItemListEndP}{\end{itemize}}

% Resume item
\renewcommand{\labelitemii}{\bulletstyle}
\renewcommand{\labelitemi}{\bulletstyle}


\newcommand{\resumeItem}[1]{
  \item\small{
    {\justifying #1 \vspace{-2pt}}
  }
}

% Entry with title, subheading, date(s), and location
\newcommand{\resumeEntryTSDL}[4]{
  \vspace{-1pt}\item[]
    \begin{tabularx}{0.97\textwidth}{X@{\hspace{60pt}}r}
      \textbf{\color{secondary}#1} & {\semibook\color{accent}\small#2} \\
    \end{tabularx}
    \begin{tabularx}{0.97\textwidth}{X@{\hspace{60pt}}r}
      \textit{\color{accent}\small#3} & \textit{\color{accent}\small#4} \\
    \end{tabularx}\vspace{-3pt}
    % \resumeItemListStart
    %     \resumeItem {\small{{\par #3 \hfill \textit{\color{accent}#4} \par}}}
    % \resumeItemListEnd\vspace{-6pt}
}

% Entry with title and date(s)
\newcommand{\resumeEntryTD}[3]{
  \vspace{-3pt}\item[]
    \begin{tabularx}{0.97\textwidth}{X@{\hspace{60pt}}r}
      \textbf{\color{secondary}#1} & {\semibook\color{accent}\small#2} \\
    \end{tabularx}  \vspace{-2pt}  
    
    {\small #3}
    \vspace{-6pt}
}

% Entry for special (skills)
\newcommand{\resumeEntryS}[2]{
  \item[]\small
    \textbf{\color{secondary}#1  }\justifying #2  \vspace{-3pt}
  
}

% Entry for special (Publications)
\newcommand{\resumeEntryP}[2]{
  \item\small{{\par \justifying #1 \hfill {\color{accent} \semibook #2} \par\vspace{-4pt}}%
  }
}

% Entry for special (Awards)
\newcommand{\resumeEntryA}[2]{
  \item\small{{\par \justifying #1 \hfill #2 \par\vspace{-8pt}}%
  }
}

% Double column header
\newcommand{\doublecol}[6]{
  \begin{tabularx}{\textwidth}{Xr}
    {
      \begin{tabular}[c]{l}
        \fontsize{35}{45}\selectfont{\color{primary}{{\heavybf\fullname}}} \\
        {\textit{\subtitle}} % You could add a subtitle here
      \end{tabular}
    } & {
      \begin{tabular}[c]{l@{\hspace{1.5em}}ll}
        {\small#4} & {\small#1} &\multirow{3}*{\includegraphics[width=0.75in]{photo.jpg}}\\
        {\small#5} & {\small#2} \\
        {\small#6} & {\small#3}
      \end{tabular}
    }
  \end{tabularx}
}

% Single column header
\newcommand{\singlecol}[5]{
  \begin{tabularx}{0.985\textwidth}{Xr}
    {
      \begin{tabular}[b]{l}
        \fontsize{35}{45}\selectfont{\color{secondary}{{\heavybf\fullname}}} \vspace{15pt}\\
        {\textit{\subtitle}}   \vspace{5pt}% You could add a subtitle here
      \end{tabular}
    } & {
      \begin{tabular}[b]{c@{\hspace{5pt}}l@{\hspace{1.5em}}l}
        {\phoneicon} &{\small#1} &\multirow{3}*{\includegraphics[width=0.75in]{photo.jpg}}\\
        {\emailicon} &{\small#2} \\
        {\websiteicon} &{\small#3} \\
        {\hometownicon} &{\small#4} \\
        {\birthdayicon} &{\small#5} 
      \end{tabular}
    }
  \end{tabularx}
}

\begin{document}
%-------------------------------------------------- BEGIN HERE --------------------------------------------------

%---------------------------------------------------- HEADER ----------------------------------------------------

\headertype{\phone}{\email}{\website}{\hometown}{\birthday} % Set the order of items here
\vspace{-10pt} % Set a negative value to push the body up, and the opposite

%-------------------------------------------------- EDUCATION --------------------------------------------------
\section{\faGraduationCap}{Education}{{\footnotesize\color{accent}\textmd{\textit{ (详细的课程成绩分析请访问\href{https://app.powerbi.com/view?r=eyJrIjoiMmI2NWZkYTUtMmZiMS00YWMxLWFmZDMtNzU3M2EwZWVhMzU2IiwidCI6IjlhYWFiZDMyLTBjNmItNGM2ZS04MTgwLTUyOTFhYjRhY2JiNCIsImMiOjZ9}{Power BI报表})}}}}

\resumeEntryStart
    \resumeEntryTSDL
    {东南大学 \button{硕士}\buttonb{985}\buttonb{211}}{20XX.09 -- 20XX.06 (expected)}
    { 移动通信国家重点实验室,电子信息{\footnotesize(通信工程)} [{\semibookit 专业排名X/XXX}],导师为\href{https://radio.seu.edu.cn/XXXXXXX/page.htm}{\semibookit XX}教授 }{南京,江苏}

    \resumeEntryTSDL
    {西安电子科技大学 \button{本科}\buttonb{211}}{20XX.09 -- 20XX.06}
    { 电子工程学院,电子信息工程 [{\semibookit  专业排名X/XXX}] }{西安,陕西}
\resumeEntryEnd


%-------------------------------------------------- RESULTS --------------------------------------------------
\section{\faPenNib}{Publications}{}

\resumeItemListStartP
    \resumeEntryP{{\semibf XXXXXX~XX}, XXXXXX~XX, XXXXXX~XX, XXXXXX~XX and XXXXXX~XX, \href{https://ieeexplore.ieee.org/document/xxxxxxxx/}{``……………………,"} {\semibookit IEEE …………}, vol.~XX, no.~XX, pp.~XXXX--XXXX, Xxx.~20XX.      }{~[SCI X作 | JCR-Qx]}
    \resumeEntryP{XXXXXX~XX, XXXXXX~XX, {\semibf XXXXXX~XX}, XXXXXX~XX and XXXXXX~XX, \href{https://ieeexplore.ieee.org/document/xxxxxxxx/}{``……………………,"} {\semibookit IEEE …………}, vol.~XX, no.~XX, pp.~XXXX--XXXX, Xxx.~20XX.     }{~[SCI X作 | JCR-Qx]}
    \resumeEntryP{{\semibf XXX}, XXX. ``XXXXXXXXXX,'' {第XX届南京地区研究生通信年会}, 20XX.}{~[一作]}
\resumeItemListEndP


%-------------------------------------------------- PROJECTS --------------------------------------------------
\section{\faBroadcastTower}{Projects}{}

\resumeEntryStart


    \resumeEntryTD
    {基于XX的XXXXXXXX联合设计}{XX校企合作项目}{{\semibf 【研究内容】}这个项目的主要研究内容是什么?}
    \resumeItemListStart
        \resumeItem {考虑到……的问题,提出了一种{\semibook ……的设计},通过……,解决了……。相关论文已发表于……{\semibookit ……}。}
        \resumeItem {考虑到……的矛盾,……,提出并搭建了一种{\semibook ……的设计},通过……,结合……,解决了……。相关论文已发表于……{\semibookit ……}。}
    \resumeItemListEnd
    \resumeEntryTD
    {基于XX的XXXXXXX设计}{XXXXXX}{{\semibf 【研究内容】}这个项目的主要研究内容是什么?}
    \resumeItemListStart
        \resumeItem {考虑到……的问题,提出了一种{\semibook ……的设计},通过……,解决了……。相关论文已发表于……{\semibookit ……}。}
    \resumeItemListEnd

\resumeEntryEnd

%-------------------------------------------------- AWARDS --------------------------------------------------
\section{\faTrophy}{Awards}{{\footnotesize\color{accent}\textmd{\textit{ (访问\href{https://levitate-qian.github.io/details/}{博客详情}了解更多,访问密码为{XXXXXXX})}}}\hfill {\small \color{secondary}\faCalendar* \hspace{3pt}} }


        \resumeItemListStartP
            \resumeEntryA{\button{荣誉} 20XX年{\semibook 研究生国家奖学金}}{20XX.11}
            \resumeEntryA{\button{荣誉} 20XX~-~20XX学年{\semibook 校级优秀研究生干部}、校级X等奖学金}{20XX.10}
            \resumeEntryA{\button{竞赛} 20XX年第XX届中国研究生数学建模竞赛X等奖}{20XX.01}
            \resumeEntryA{\button{荣誉} 20XX年毕业生校级X等奖学金、{\semibook 校级优秀毕业生}}{20XX.06}
            \resumeEntryA{\button{荣誉} 20XX~-~20XX学年{\semibook 国家奖学金、校级优秀学生标兵}}{20XX.11}
            \resumeEntryA{\button{竞赛} 20XX年{\semibook 国际大学生数学建模竞赛 XXXXXXXXXXXX}}{20XX.04}
        \resumeItemListEndP



%-------------------------------------------------- PROGRAMMING SKILLS --------------------------------------------------
\section{\faBuffer}{Skills}{}
\resumeEntryStart
    \resumeEntryS{英语水平} {{\semibook CET-6 优秀 (XXX分)。}用于支撑英语水平……}
    \resumeEntryS{编程水平} {熟悉{\semibook ……编程}。熟悉……框架。了解……编程。}
    \resumeEntryS{专业水平} {熟悉……,了解……。参与 …… 相关综述论文撰写,主要负责……调研撰写。}
    \resumeEntryS{XXXX} {{\semibook 熟悉\LaTeX,推文排版。}在哪儿做什么,有什么收获。(用于支撑该技能的案例)}
\resumeEntryEnd



\end{document}
